\chapter{Conception}
Um am Ende der Entwicklung sichergehen zu können, dass das Programm die gewünschte Funktionalität bietet, ist es wichtig zu Beginn die Anforderungen zu definieren und das Projekt nach Abschluss damit zu evaluieren. 
\section{Anforderungsanalyse}
%TODO Koen: Wenn dir noch was einfällt, bitte ergänzen (und zwar direkt in den Text)
Der Fokus dieses Projektes liegt auf die Darstellung und anschauliche Demonstration des Davis Putman Algorithmus anhand des N Damen Problems. Um dieses bestmöglich zu ermöglichen, müssen folgende Anforderungen erfüllt werden. 
\begin{itemize}
\item Die Größe von N für das N Damen Problem muss über eine Benutzeroberfläche veränderbar sein
\item Bei jedem Versuch ein bestimmtes N Damen Problem zu lösen, sollen unterschiedliche Rechenwege und somit auch Lösungen entstehen. Jedoch muss aber gewährleistet sein, dass es jederzeit die Möglichkeit gibt, Ergebnisse replizierbar wiederzuzeigen. 
\item Der Prozess zur Lösung des N Damen Problems muss einerseits auf einem Schachbrett visuell in einzelnen Schritten gezeigt und andererseits als einzelne Kalkulationsschritte wiedergegeben werden. Dabei muss zur Wahl stehen, ob der Nutzer Macro oder Micro Schritte betrachten möchte. Je nachdem welcher Schritt ausgewählt wurde, ändert sich die Größe der Zwischenschritte.
\item Die Simualtion muss pausierbar sein.
\end{itemize}
Am Ende der Implementierungsphase wird das Programm anhand dieser Anforderungen gemessen und je nachdem als Erfolg oder eben als Nichterfolg deklariert. 

\section{Architecture Sketch}

\section{Layout Design}