% -------------------------------------------------------
% Daten für die Arbeit
% Wenn hier alles korrekt eingetragen wurde, wird das Titelblatt
% automatisch generiert. D.h. die Datei titelblatt.tex muss nicht mehr
% angepasst werden.

\newcommand{\dhbwsprache}{en} % de oder en für Deutsch oder Englisch
                              % Für korrekt sortierte Literatureinträge, noch preambel.tex anpassen

% Titel der Arbeit auf Deutsch
\newcommand{\dhbwtitelde}{Deep Learning Based Object Classification on Edge Devices}

% Titel der Arbeit auf Englisch
\newcommand{\dhbwtitelen}{Deep Learning Based Object Classification on Edge Devices}

% Weitere Informationen zur Arbeit
\newcommand{\dhbwort}{Mannheim}    % Ort
\newcommand{\dhbwautorvname}{Koen} % Vorname(n)
\newcommand{\dhbwautornname}{Logmann} % Nachname(n)
\newcommand{\dhbwcoautorvname}{Jessica} % Vorname(n)
\newcommand{\dhbwcoautornname}{Roth} % Nachname(n)
\newcommand{\dhbwdatum}{\today} % Datum der Abgabe
\newcommand{\dhbwjahr}{2019} % Jahr der Abgabe
\newcommand{\dhbwfirma}{SAP SE, Walldorf} % Firma bei der die Arbeit durchgeführt wurde
\newcommand{\dhbwbetreuer}{Prof. Dr. Ivo Wolf, Hochschule Mannheim} % Betreuer an der Hochschule
\newcommand{\dhbwzweitkorrektor}{Dr. Julien Vayssiere, SAP SE} % Betreuer im Unternehmen oder Zweitkorrektor
\newcommand{\dhbwfakultaet}{I} % I für Informatik
\newcommand{\dhbwstudiengang}{AI} % IB IMB UIB IM MTB

% Zustimmung zur Veröffentlichung
\setboolean{dhbwpublizieren}{true}   % Einer Veröffentlichung wird zugestimmt
\setboolean{dhbwsperrvermerk}{false} % Die Arbeit hat keinen Sperrvermerk

% -------------------------------------------------------
% Abstract

% Kurze (maximal halbseitige) Beschreibung, worum es in der Arbeit geht auf Deutsch
\newcommand{\dhbwabstractde}{Die Hauptthematik der vorliegenden Masterarbeit umfasst den Vergleich zwischen Edge Devices und Cloud Services für die Objektklassifizierung mittels Deep Learning. Die grundlegende Fragestellung dabei ist, welches die beste Arbeitsumgebung dafür bildet. Um diese Frage zu erörtern ist es notwendig, zu Beginn geeignete Netzwerke zu finden und diese auf den Probanten optimal zu installieren und zu trainieren. Mittels der experimentellen Methode werden folgend alle Netzwerke auf den Edge Devices und auf den Cloud Services getestet und evaluiert. Die daraus resultierenden Ergebnisse werden untereinander verglichen und mittels einer gewichteten Entscheidungsmatrix bewertet. Entgegen dem aktuellen Trend zu den Cloud Services hin, zeigten die Ergebnisse des Vergleichs eine Daseinsbereichtigung der Edge-Devices und eine Empfehlung hin zu diesen aufgrund mehrerer serverseitiger Probleme im Umgang mit den Cloud Services. }

% Kurze (maximal halbseitige) Beschreibung, worum es in der Arbeit geht auf Englisch

\newcommand{\dhbwabstracten}{The main topic of this master thesis is the comparison between Edge Devices and Cloud Services for object classification using Deep Learning. The basic question is which of the two variants represents the best working environment for this. In order to discuss this question, it is necessary to find suitable networks at the beginning and to install and train them optimally on the test equipment. Using the experimental method, all networks on the edge devices and cloud services will be tested and evaluated. The results are compared with each other and evaluated using a weighted decision matrix. Contrary to the current trend towards cloud services, the results of the comparison showed that the Edge devices were in the same category and a recommendation for them due to several server-side problems in dealing with cloud services for specific applications.}

