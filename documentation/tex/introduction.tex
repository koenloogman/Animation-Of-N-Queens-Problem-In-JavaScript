% !TeX root = ./documentation.tex

\chapter{Introduction}
\section{Motivation}
\section{Objective of the work}
Die Aufgabenstellung dieser Studienarbeit handelt von der Animation des Davis Putman Algorithmus anhand des N Damen Problems mit der Hilfe von Javascript. Dabei werden mehrere Kriterien definiert, die diese Animation erfüllen muss, da sie vor allem zur Visualisierung und zur genaueren anschaulichen Verdeutlichung beziehungsweise Erklärung des Algorithmus dient. Die genauen Anforderungen werden im Laufe dieser Arbeit genauer spezifiziert und am Ende am finalen Produkt evaluiert. Der Hauptteil dieser Arbeit handelt deshalb von der Konzeption und der Implementierung dieser Aufgabenstellung.
%TODO Koen: Fällt dir noch etwas dazu ein?
\section{Structure of the report}
Um die Navigation durch diese Arbeit zu erleichtern, wird im folgenden Abschnitt einen kurzen Blick in die Struktur dieser Arbeit geworfen. 
\\
Im nächsten Kapitel werden die mathematischen Grundlagen zum Davis Putman Algorithmus und des N Damen Problems geschaffen. Dabei handelt es sich vor allem um die logischen Zusammenhänge und der mathematisch korrekten Definition des zu lösenden Problems. 
\\
Darauffolgend wird im dritten Kapitel ein allgemeiner Überblick zu den verwendeten Bibliotheken und Technologien gegeben und Hintergrundinformationen geliefert, aus welchen Gründen bestimmte Technologien bevorzugt beziehungsweise ausgewählt wurden.
\\
Im vierten Abschnitt dieser Arbeit wird die Konzeption des Programms genauer betrachtet. Vor allem handelt es von der allgemeinen Architektur und dem erstellten Layout Design, das als Vorlage für die Implementierung der Benutzeroberfläche dient. Eingeleitet wird das Kapitel durch das Spezifizieren der Anforderungen für das Programm, die eine zentrale Rolle in der Evaluation am Ende spielen.
\\
Im fünften Kapitel wird nun Schritt für Schritt durch die fertige Implementierung geleitet und der Aufbau des Programms genauer erläutert. Dabei wird durch die unterschiedlichen Komponenten und Klassen geführt und dabei ihre Funktion genauer erläuert.
\\
Das letzte Kapitel wird mit der Evaluation der Anforderungen begonnen. Dies dient vor allem dazu, um zu schauen, ob das Programm alle Kriterien erfüllt und zu Ende als erfolgreich erfüllt angesehen werden kann. Danach folgt ein zusammenfassendes Fazit und ein Ausblick, wie das Programm zukünftig weiterentwickelt oder anders eingesetzt werden könnte.