% !TeX root = ./documentation.tex

\chapter{Introduction}
\label{ch:introduction}

\section{Motivation}
\label{sec:introMotivation}
% cSpell:disable
%TODO Koen: Fällt dir was besseres ein?
%Der Davis Putman Algorithmus ist ein wichtiges Werkzeug zur Lösung von bekannten mathematischen Problemen wie beispielsweise das N Damen Problem. Dies ist auch der Grund, weshalb dieser oftmals zum Erklären von aussagelogischen Prinzipien verwendet wird. Oftmals ist es aber schwierig nachzuvollziehen, wie der ALgorithmus aussagelogische Probleme löst, da es allein durch das Betrachten der Rechenwege des Algorithmus nur schwer zu verstehen ist.
%\\
%Wenn es die Möglichkeit gäbe, die Lösung des N Damen Problems Schritt für Schritt zu verfolgen und dies visuell veständlich angezeigt zu bekommen, dann würde dies viele Verständnisprobleme klären und den Lernprozess deutlich beschleunigen. Dies ist auch der Grund, weshalb dieses Programm zur Visualisieren des Davis Putman Algorithmus so wichtig ist. Es kann zu vielen Zwecken wie beispielsweise der Unterstützung von Lehrveranstaltungen verwendet werden und hilft damit vielen Interessierten einen einfachen Einstieg in diesen komplexen Algorithmus.
% cSpell:enable
The \textit{Davis-Putnam algorithm} is an important tool for solving important problems from propositional logic such as the \textit{N-Queens Problem}. This is also the reason why it is regularly used to explain propositional logic principles. However, it is often difficult to understand how the algorithm finds possible solutions for propositional logic problems, as it is difficult to understand by just looking at the algorithm's computational paths.

If there would be the possibility to observe the \textit{Davis-Putnam algorithm} trying to find a solution for the \textit{N-Queens Problem} step by step with a visual representation of the current state, then this would clarify many understanding problems and enhance the learning process. This is also the reason why the program developed in this thesis is so important for visualizing the Davis-Putnam algorithm. It can be used for many purposes, such as supporting lectures, and thus helps many interested people to easily get started with this complex algorithm.

\section{Objective of the work}
\label{sec:introObjective}
% cSpell:disable
%TODO Koen: Fällt dir noch etwas dazu ein?
%Die Aufgabenstellung dieser Studienarbeit handelt von der Animation des Davis Putman Algorithmus anhand des N Damen Problems mit der Hilfe von Javascript. Dabei werden mehrere Kriterien definiert, die diese Animation erfüllen muss, da sie vor allem zur Visualisierung und zur genaueren anschaulichen Verdeutlichung beziehungsweise Erklärung des Algorithmus dient. Die genauen Anforderungen werden im Laufe dieser Arbeit genauer spezifiziert und am Ende am finalen Produkt evaluiert. Der Hauptteil dieser Arbeit handelt deshalb von der Konzeption und der Implementierung dieser Aufgabenstellung.
% cSpell:enable
The task of this student research project is the animation of the \textit{Davis-Putnam algorithm} on the basis of the \textit{N-Queens Problem} with the help of JavaScript. Several criteria are defined, which this animation has to fulfill, since it serves above all for the visualization and for the more exact descriptive clarification and/or explanation of the algorithm. The exact requirements are specified more precisely in the course of this work and evaluated at the end of the final product. The main part of this work therefore deals with the conception and implementation of this task.

\section{Structure of the report}
\label{sec:introStructure}
% cSpell:disable
%Um die Navigation durch diese Arbeit zu erleichtern, wird im folgenden Abschnitt einen kurzen Blick in die Struktur dieser Arbeit geworfen. 
%\\
%Im nächsten Kapitel werden die mathematischen Grundlagen zum Davis Putman Algorithmus und des N Damen Problems geschaffen. Dabei handelt es sich vor allem um die logischen Zusammenhänge und der mathematisch korrekten Definition des zu lösenden Problems. 
%\\
%Darauffolgend wird im dritten Kapitel ein allgemeiner Überblick zu den verwendeten Bibliotheken und Technologien gegeben und Hintergrundinformationen geliefert, aus welchen Gründen bestimmte Technologien bevorzugt beziehungsweise ausgewählt wurden.
%\\
%Im vierten Abschnitt dieser Arbeit wird die Konzeption des Programms genauer betrachtet. Vor allem handelt es von der allgemeinen Architektur und dem erstellten Layout Design, das als Vorlage für die Implementierung der Benutzeroberfläche dient. Eingeleitet wird das Kapitel durch das Spezifizieren der Anforderungen für das Programm, die eine zentrale Rolle in der Evaluation am Ende spielen.
%\\
%Im fünften Kapitel wird nun Schritt für Schritt durch die fertige Implementierung geleitet und der Aufbau des Programms genauer erläutert. Dabei wird durch die unterschiedlichen Komponenten und Klassen geführt und dabei ihre Funktion genauer erläuert.
%\\
%Das letzte Kapitel wird mit der Evaluation der Anforderungen begonnen. Dies dient vor allem dazu, um zu schauen, ob das Programm alle Kriterien erfüllt und zu Ende als erfolgreich erfüllt angesehen werden kann. Danach folgt ein zusammenfassendes Fazit und ein Ausblick, wie das Programm zukünftig weiterentwickelt oder anders eingesetzt werden könnte.
% cSpell:enable
In order to facilitate navigation through this work, the following section takes a brief look at the structure of this work.

The next chapter explains the scientific basics of the \textit{Davis-Putnam algorithm} and the \textit{N-Queens Problem}. These are mainly about the logical correlations and the mathematically correct definition of the problem to be solved.

The third chapter then gives a general overview of the libraries and technologies used and provides background information on the reasons why certain technologies were preferred or selected.

The fourth section of this paper takes a closer look at the design of the program. It mainly deals with the general architecture and the created layout design, which serves as a template for the implementation of the user interface. The chapter is introduced by specifying the requirements for the program, which are mandatory for the evaluation at the end.

In the fifth chapter, step by step you will be guided through the implementation and the structure of the program will be explained in more detail. This chapter guides through the different components and classes and explains their roles in more detail.

The last chapter starts with the evaluation of the requirements. This is mainly to see if the program meets all the criteria and can be considered successful at the end. This is followed by a summary and an outlook on how the program or its components could be further developed or used in other ways in the future.