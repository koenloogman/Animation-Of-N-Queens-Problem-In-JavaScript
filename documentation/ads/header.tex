%!TEX root = ../dokumentation.tex

%
% Nahezu alle Einstellungen koennen hier getaetigt werden
%

\RequirePackage[l2tabu, orthodox]{nag}	% weist in Commandozeile bzw. log auf veraltete LaTeX Syntax hin

\documentclass[%
	pdftex,	
	oneside,			% Einseitiger Druck.
	12pt,				% Schriftgroesse
	parskip=half,		% Halbe Zeile Abstand zwischen Absätzen.
%	topmargin = 10pt,	% Abstand Seitenrand (Std:1in) zu Kopfzeile [laut log: unused]
	headheight = 14pt,	% Höhe der Kopfzeile
%	headsep = 30pt,	% Abstand zwischen Kopfzeile und Text Body  [laut log: unused]
	headsepline,		% Linie nach Kopfzeile.
	footsepline,		% Linie vor Fusszeile.
	footheight = 16pt,	% Höhe der Fusszeile
	abstracton,		% Abstract Überschriften
	DIV=calc,		% Satzspiegel berechnen
	BCOR=8mm,		% Bindekorrektur links: 8mm
	headinclude=false,	% Kopfzeile nicht in den Satzspiegel einbeziehen
	footinclude=false,	% Fußzeile nicht in den Satzspiegel einbeziehen
	listof=totoc,		% Abbildungs-/ Tabellenverzeichnis im Inhaltsverzeichnis darstellen
	toc=bibliography,   % Literaturverzeichnis im Inhaltsverzeichnis darstellen
]{scrreprt}	% Koma-Script report-Klasse, fuer laengere Bachelorarbeiten alternativ auch: scrbook scrreprt

% Einstellungen laden
\usepackage{xstring}
\usepackage[utf8]{inputenc}
\usepackage[T1]{fontenc}
\usepackage{amsmath}
\usepackage{amsthm}
\usepackage{amsbsy}
\usepackage{amssymb} 
\usepackage{tabularx}
\usepackage{multirow}
\usepackage{booktabs}

\newcommand{\einstellung}[1]{%
  \expandafter\newcommand\csname #1\endcsname{}
  \expandafter\newcommand\csname setze#1\endcsname[1]{\expandafter\renewcommand\csname#1\endcsname{##1}}
}
\newcommand{\langstr}[1]{\einstellung{lang#1}}

\einstellung{matrikelnr}
\einstellung{titel}
\einstellung{kurs}
\einstellung{datumAbgabe}
\einstellung{firma}
\einstellung{firmenort}
\einstellung{abgabeort}
\einstellung{studiengang}
\einstellung{dhbw}
\einstellung{abteilung}
\einstellung{betreuer}
\einstellung{spaceline}
\einstellung{zeitraum}
\einstellung{arbeit}
\einstellung{autor}
\einstellung{sprache}
\einstellung{schriftart}
\einstellung{seitenrand}
\einstellung{kapitelabstand}
\einstellung{spaltenabstand}
\einstellung{zeilenabstand}
\einstellung{zitierstil}
 % verfügbare Einstellungen
\input{ads/einstellungen} % lese Einstellungen

\input{lang/strings} % verfügbare Strings
\input{lang/\sprache} % Übersetzung einlesen

% Einstellung der Sprache des Paketes Babel und der Verzeichnisüberschriften
\iflang{de}{\usepackage[english, ngerman]{babel}}
\iflang{en}{\usepackage[ngerman, english]{babel}} 

%%%%%%% Package Includes %%%%%%%

\usepackage[margin=\seitenrand,foot=1cm]{geometry}	% Seitenränder und Abstände
\usepackage[activate]{microtype} %Zeilenumbruch und mehr
\usepackage[onehalfspacing]{setspace}
\usepackage{makeidx}
\usepackage[autostyle=true,german=quotes]{csquotes}
\usepackage{longtable}
\usepackage{enumitem}	% mehr Optionen bei Aufzählungen
\usepackage{graphicx}
\usepackage{pdfpages}   % zum Einbinden von PDFs
\usepackage{xcolor}
%\usepackage[dvipsnames]{xcolor} 	% für HTML-Notation
\usepackage{float}
\usepackage{array}
\usepackage{calc}		% zum Rechnen (Bildtabelle in Deckblatt)
\usepackage[right]{eurosym}
\usepackage{wrapfig}
\usepackage{pgffor} % für automatische Kapiteldateieinbindung
\usepackage[perpage, hang, multiple, stable]{footmisc} % Fussnoten
\usepackage[printonlyused]{acronym} % falls gewünscht kann die Option footnote eingefügt werden, dann wird die Erklärung nicht inline sondern in einer Fußnote dargestellt
\usepackage{listings}
\usepackage{minted}
\usepackage{fancyhdr}
%\usepackage{slashbox}
\usepackage[hang,font={sf,footnotesize},labelfont={footnotesize,bf}]{caption} % Beschriftungen
\usepackage{subfigure}				% Mehrere Bilder in einem


\newfont{\chess}{chess20}
\newfont{\bigchess}{chess30}
\newcommand{\chf}{\baselineskip20pt\lineskip0pt\chess}

% Notizen. Einsatz mit \todo{Notiz} oder \todo[inline]{Notiz}. 
\usepackage[obeyFinal,backgroundcolor=yellow,linecolor=black]{todonotes}
% Alle Notizen ausblenden mit der Option "final" in \documentclass[...] oder durch das auskommentieren folgender Zeile
% \usepackage[disable]{todonotes}

% Kommentarumgebung. Einsatz mit \comment{}. Alle Kommentare ausblenden mit dem Auskommentieren der folgenden und dem aktivieren der nächsten Zeile.
\newcommand{\comment}[1]{\par {\bfseries \color{blue} #1 \par}} %Kommentar anzeigen
% \newcommand{\comment}[1]{} %Kommentar ausblenden

% Author mit komplettem Name zitieren
\newrobustcmd*{\citefirstlastauthor}{\AtNextCite{\DeclareNameAlias{labelname}{first-last}}\citeauthor}


%%%%%% Configuration %%%%%

%% Anwenden der Einstellungen

\usepackage{\schriftart}
\ladefarben{}

% Titel, Autor und Datum
\title{\titel}
\author{\autor}
\date{\datum}

% PDF Einstellungen
\usepackage[%
	pdftitle={\titel},
	pdfauthor={\autor},
	pdfsubject={\arbeit},
	pdfcreator={pdflatex, LaTeX with KOMA-Script},
	pdfpagemode=UseOutlines, 		% Beim Oeffnen Inhaltsverzeichnis anzeigen
	pdfdisplaydoctitle=true, 		% Dokumenttitel statt Dateiname anzeigen.
	pdflang={\sprache}, 			% Sprache des Dokuments.
]{hyperref}

% (Farb-)einstellungen für die Links im PDF
\hypersetup{%
	colorlinks=true, 		% Aktivieren von farbigen Links im Dokument
	linkcolor=LinkColor, 	% Farbe festlegen
	citecolor=LinkColor,
	filecolor=LinkColor,
	menucolor=LinkColor,
	urlcolor=LinkColor,
	linktocpage=false, 		% Nicht der Text sondern die Seitenzahlen in Verzeichnissen klickbar
	bookmarksnumbered=true 	% Überschriftsnummerierung im PDF Inhalt anzeigen.
}
% Workaround um Fehler in Hyperref, muss hier stehen bleiben
\usepackage{bookmark} %nur ein latex-Durchlauf für die Aktualisierung von Verzeichnissen nötig

% Schriftart in Captions etwas kleiner
\addtokomafont{caption}{\small}


% Literaturverweise (sowohl deutsch als auch englisch)
\iflang{de}{%
\usepackage[
	backend=biber,		% empfohlen. Falls biber Probleme macht: bibtex
	bibwarn=true,
	bibencoding=utf8,	% wenn .bib in utf8, sonst ascii
	sortlocale=de_DE,
	style=\zitierstil,
]{biblatex}
}
\iflang{en}{%
\usepackage[
	backend=biber,		% empfohlen. Falls biber Probleme macht: bibtex
	bibwarn=true,
	bibencoding=utf8,	% wenn .bib in utf8, sonst ascii
	sortlocale=en_US,
	style=\zitierstil,
]{biblatex}
}

\ladeliteratur{}

% Glossar
\usepackage[nonumberlist,toc]{glossaries}

%%%%%% Additional settings %%%%%%

% Hurenkinder und Schusterjungen verhindern
% http://projekte.dante.de/DanteFAQ/Silbentrennung
\clubpenalty = 10000 % schließt Schusterjungen aus (Seitenumbruch nach der ersten Zeile eines neuen Absatzes)
\widowpenalty = 10000 % schließt Hurenkinder aus (die letzte Zeile eines Absatzes steht auf einer neuen Seite)
\displaywidowpenalty=10000

% Bildpfad
\graphicspath{{images/}}

% Einige häufig verwendete Sprachen
%\lstloadlanguages{PHP,Python,Java,C,C++,bash}
%\listingsettings{}

% Farben Code-higlighten
\definecolor{mygreen}{rgb}{0,0.6,0}
\definecolor{mygray}{rgb}{0.5,0.5,0.5}
\definecolor{mydarkgray}{rgb}{0.3,0.3,0.3}
\definecolor{mymauve}{rgb}{0.58,0,0.82}
\definecolor{mypurple}{RGB}{128,0,192}
\definecolor{mymagenta}{rgb}{1,0,1}
\definecolor{myClassBlue}{RGB}{43,145,175}
\definecolor{myClassBrown}{RGB}{152,104,1}
\definecolor{myStringRed}{RGB}{163,21,21}
\definecolor{myCommentGreen}{RGB}{0,128,0}
\definecolor{mymagenta}{rgb}{1,0,1}


% CSS
\lstdefinelanguage{CSS}{
  keywords={color,background-image:,margin,padding,font,weight,display,position, %
  top,left,right,bottom,list,style,border,size,white,space,min,width, transition:,
  transform:, transition-property, transition-duration, transition-timing-function, repeate, row, translateX, calc, var}, 
  sensitive=true,
  morecomment=[l]{//},
  morecomment=[s]{/*}{*/},
  morestring=[b]',
  morestring=[b]",
  alsoletter={:.-}
  %alsodigit={-, px, fr},
  %alsodigit={-}
}

% Markdown
\lstdefinelanguage{MarkDown}{
  morekeywords={},
  morecomment=[s]{/*}{*/},
  morecomment=[l]//,
  morestring=[b]",
  morestring=[b]'
}

% JavaScript
\lstdefinelanguage{JavaScript}{
  morekeywords={typeof, new, true, false, catch, function, return, null, catch, switch, var, if, in, while, do, else, case, break, for, ease, const},
  morecomment=[s]{/*}{*/},
  morecomment=[l]//,
  morestring=[b]",
  morestring=[b]'
}

% Pug JS
\lstdefinelanguage{PugJs}{
  morekeywords={html, body, h1, p},
  morecomment=[s]{/*}{*/},
  morecomment=[l]//,
  morestring=[b]",
  morestring=[b]'
}

% HTML5 + Liquid
\lstdefinelanguage{HTML5}{
  language=html,
  sensitive=true,   
  alsoletter={<>=-},    
  morecomment=[s]{<!-}{-->},
  tag=[s],
  otherkeywords={
  % General
  >,
  % Standard tags
    <!DOCTYPE,
  </html, <html, <head, <title, </title, <style, </style, <link, </head, <meta, />,
    % body
    </body, <body,
    % Divs
    </div, <div, </div>, 
    % Paragraphs
    </p, <p, </p>,
    % scripts
    </script, <script,
  % More tags...
  <canvas, /canvas>, <svg, <rect, <animateTransform, </rect>, </svg>, <video, <source, <iframe, </iframe>, </video>, <image, </image>, <header, </header, <article, </article, <h1, </h1, endfor, include
  },
  ndkeywords={
  % General
  =,
  % HTML attributes
  charset=, src=, id=, width=, height=, style=, type=, rel=, href=,
  % SVG attributes
  fill=, attributeName=, begin=, dur=, from=, to=, poster=, controls=, x=, y=, repeatCount=, xlink:href=,
  % properties
  margin:, padding:, background-image:, border:, top:, left:, position:, width:, height:, margin-top:, margin-bottom:, font-size:, line-height:,
    % CSS3 properties
  transform:, -moz-transform:, -webkit-transform:,
  animation:, -webkit-animation:,
  transition:,  transition-duration:, transition-property:, transition-timing-function:,
  }
}

% Vorlage 
\lstdefinestyle{htmlcssjs} {%
  % General design
%  backgroundcolor=\color{editorGray},
  basicstyle={\footnotesize\ttfamily},   
  frame=b,
  % line-numbers
  xleftmargin={0.75cm},
  numbers=left,
  stepnumber=1,
  firstnumber=1,
  numberfirstline=true, 
  % Code design
  identifierstyle=\color{black},
  keywordstyle=\color{blue}\bfseries,
  ndkeywordstyle=\color{myClassBlue}\bfseries,
  stringstyle=\color{myStringRed}\ttfamily,
  commentstyle=\color{mygreen}\ttfamily,
  % Code
  language=HTML5,
  alsolanguage=JavaScript,
  alsodigit={.:;},  
  tabsize=2,
  showtabs=false,
  showspaces=false,
  showstringspaces=false,
  extendedchars=true,
  breaklines=true,
  % German umlauts
  literate=%
  {Ö}{{\"O}}1
  {Ä}{{\"A}}1
  {Ü}{{\"U}}1
  {ß}{{\ss}}1
  {ü}{{\"u}}1
  {ä}{{\"a}}1
  {ö}{{\"o}}1
}

%Formatieren von SCSS Code
\lstdefinestyle{SCSS}{
	% General Design
	% Background
	backgroundcolor=\color{white}, % 	
	basicstyle=\ttfamily\scriptsize,	
	%frame=single, % 
	frame=trBL,
	framexleftmargin=5mm,
	% line-numbers
	numbers=left, % 
	numberstyle=\tiny\color{mygray}, % 
	% Code-highlighting	
	identifierstyle=\color{blue}, % 
	keywordstyle=\color{blue}, %  
	stringstyle=\color{myStringRed}, % 
    commentstyle=\color{mygreen}, % 
    showspaces=false, % 
    showstringspaces=false, %    
	% Code
	language=CSS,
	tabsize=2, % 
	extendedchars=true, % 
	breaklines=true, % 
	%float=hbp,% 
    %columns=flexible, %     
    breakautoindent=true, % 
    captionpos=b, %
    sensitive=false, 
	morecomment=[l]{//}, 
	morecomment=[s]{/*}{*/}, 
	morestring=[b]',
	%Hinzufügen von Klassen Namen
	emph=[1]{.grid, .grid__title, .grid__hamburger, .grid__collection, .grid__main, .grid__nav, .grid__links, .sidebar, .sidebar__label, .has-open-sidebar, .sidebar__items, .sidebar__item},
	emphstyle=[1]{\color{myClassBlue}},
	% Hinzufügen von zu markierenden Enums und defines	
	emph=[2]{--sidebar-width, \$Color__gray--100, --right-space, \$links-height},
	emphstyle=[2]{\color{mypurple}},
	%	
	emph=[3]{repeat, minmax},
	emphstyle=[3]{\color{mydarkgray}},
	% Sonstiges
	emph=[4]{px, fr, .5s, grid},
	emphstyle=[4]{\color{black}},	
	literate =%
	{Ä}{{\"A}}1
	{Ö}{{\"O}}1
	{Ü}{{\"U}}1
	{ä}{{\"a}}1
	{ö}{{\"o}}1
	{ü}{{\"u}}1
	{ß}{{\ss}}1
}


%Formatieren von Markdown Code
\lstdefinestyle{MARKDOWN}{
	% General Design
	% Background
	backgroundcolor=\color{white}, % 	
	basicstyle=\ttfamily\scriptsize,	
	%frame=single, % 
	frame=trBL,
	framexleftmargin=5mm,
	% line-numbers
	numbers=left, % 
	numberstyle=\tiny\color{mygray}, % 
	% Code-highlighting	
	%identifierstyle=\color{red}, % 
	keywordstyle=\color{blue},
	stringstyle=\color{myStringRed}, % 
    commentstyle=\color{mygreen}, % 
    showspaces=false, % 
    showstringspaces=false, %    
	% Code
	language=MarkDown,
	tabsize=2, % 
	extendedchars=true, % 
	breaklines=true, % 
	%float=hbp,% 
    %columns=flexible, %     
    breakautoindent=true, % 
    captionpos=b, %
    sensitive=false, 
	morecomment=[l]{//}, 
	morecomment=[s]{/*}{*/}, 
	morestring=[b]',
	% Klassen/Modul Namen 
	emph=[1]{layout, title, date, categories, author, image},
	emphstyle=[1]{\color{myClassBlue}},	
	% Anweißungen
%	emph=[2]{--sidebar-width, },
%	emphstyle=[2]{\color{mypurple}},
	% Attribute	
	emph=[3]{Inhalt},
	emphstyle=[3]{\color{mydarkgray}},
	% Zusätzliche keywords
%	emph=[4]{new},
%	emphstyle=[4]{\color{blue}},		
	literate =%
	{Ä}{{\"A}}1
	{Ö}{{\"O}}1
	{Ü}{{\"U}}1
	{ä}{{\"a}}1
	{ö}{{\"o}}1
	{ü}{{\"u}}1
	{ß}{{\ss}}1
}


%Formatieren von pug.js Code
\lstdefinestyle{PUGJS}{
	% General Design
	% Background
	backgroundcolor=\color{white}, % 	
	basicstyle=\ttfamily\scriptsize,	
	%frame=single, % 
	frame=trBL,
	framexleftmargin=5mm,
	% line-numbers
	numbers=left, % 
	numberstyle=\tiny\color{mygray}, % 
	% Code-highlighting	
	%identifierstyle=\color{red}, % 
	keywordstyle=\color{blue},
	stringstyle=\color{myStringRed}, % 
    commentstyle=\color{mygreen}, % 
    showspaces=false, % 
    showstringspaces=false, %    
	% Code
	language=PugJs,
	tabsize=2, % 
	extendedchars=true, % 
	breaklines=true, % 
	%float=hbp,% 
    %columns=flexible, %     
    breakautoindent=true, % 
    captionpos=b, %
    sensitive=false, 
	morecomment=[l]{//}, 
	morecomment=[s]{/*}{*/}, 
	morestring=[b]',
	% Klassen/Modul Namen 
%	emph=[1]{},
%	emphstyle=[1]{\color{myClassBlue}},	
	% Anweißungen
	emph=[2]{include },
	emphstyle=[2]{\color{mypurple}},
	% Attribute/Methoden
	emph=[3]{doctype},
	emphstyle=[3]{\color{mydarkgray}},
	% Zusätzliche keywords
%	emph=[4]{new},
%	emphstyle=[4]{\color{blue}},		
	literate =%
	{Ä}{{\"A}}1
	{Ö}{{\"O}}1
	{Ü}{{\"U}}1
	{ä}{{\"a}}1
	{ö}{{\"o}}1
	{ü}{{\"u}}1
	{ß}{{\ss}}1
}

%Formatieren von Bash Code
\lstdefinestyle{BASH}{
	% General Design
	% Background
	backgroundcolor=\color{white}, % 	
	basicstyle=\ttfamily\scriptsize,	
	%frame=single, % 
	frame=trBL,
	framexleftmargin=5mm,
	% line-numbers
	numbers=left, % 
	numberstyle=\tiny\color{mygray}, % 
	% Code-highlighting	
	%identifierstyle=\color{red}, % 
	keywordstyle=\color{blue},
	stringstyle=\color{myStringRed}, % 
    commentstyle=\color{mygreen}, % 
    showspaces=false, % 
    showstringspaces=false, %    
	% Code
	language=Bash,
	tabsize=2, % 
	extendedchars=true, % 
	breaklines=true, % 
	%float=hbp,% 
    %columns=flexible, %     
    breakautoindent=true, % 
    captionpos=b, %
    sensitive=false, 
	morecomment=[l]{//}, 
	morecomment=[s]{/*}{*/}, 
	morestring=[b]',
	% Klassen/Modul Namen 
%	emph=[1]{ghost, npm, jekyll, gem},
%	emphstyle=[1]{\color{myClassBlue}},	
	% Anweißungen
%	emph=[2]{--sidebar-width, },
%	emphstyle=[2]{\color{mypurple}},
	% Attribute	
%	emph=[3]{hamburger, title, search, grid},
%	emphstyle=[3]{\color{mydarkgray}},
	% Zusätzliche keywords
%	emph=[4]{new},
%	emphstyle=[4]{\color{blue}},		
	literate =%
	{Ä}{{\"A}}1
	{Ö}{{\"O}}1
	{Ü}{{\"U}}1
	{ä}{{\"a}}1
	{ö}{{\"o}}1
	{ü}{{\"u}}1
	{ß}{{\ss}}1
}

%Formatieren von HTML Code mit Liquid
\lstdefinestyle{HTML}{
	% General Design
	% Background
	backgroundcolor=\color{white}, % 	
	basicstyle=\ttfamily\scriptsize,	
	%frame=single, % 
	frame=trBL,
	framexleftmargin=5mm,
	% line-numbers
	numbers=left, % 
	numberstyle=\tiny\color{mygray}, % 
	% Code-highlighting	
	%identifierstyle=\color{red}, % 
	keywordstyle=\color{blue}, 
	stringstyle=\color{myStringRed}, % 
    commentstyle=\color{mygreen}, % 
    showspaces=false, % 
    showstringspaces=false, %    
	% Code
	language=HTML5,
	tabsize=2, % 
	extendedchars=true, % 
	breaklines=true, % 
	%float=hbp,% 
    %columns=flexible, %     
    breakautoindent=true, % 
    captionpos=b, %
    sensitive=false,  
	morecomment=[s]{<!--}{-->}, 
	morestring=[b]',
	% Klassen Namen aus Liquid
	emph=[1]{body, page, post, site, article, footer, links, header, sidebar, head, navigation, title},
	emphstyle=[1]{\color{myClassBlue}},
	% Anweisungen in Liquid
	emph=[2]{for, endfor, in, include },
	emphstyle=[2]{\color{mypurple}},
	%Attribute von Liquid
	emph=[3]{content, html, posts},
	emphstyle=[3]{\color{mydarkgray}},
	literate =%
	{Ä}{{\"A}}1
	{Ö}{{\"O}}1
	{Ü}{{\"U}}1
	{ä}{{\"a}}1
	{ö}{{\"o}}1
	{ü}{{\"u}}1
	{ß}{{\ss}}1
}

%Formatieren von HTML Code mit Liquid zweite variante wegen doppeltem Name als Klasse und Attribut
\lstdefinestyle{HTML2}{
	% General Design
	% Background
	backgroundcolor=\color{white}, % 	
	basicstyle=\ttfamily\scriptsize,	
	%frame=single, % 
	frame=trBL,
	framexleftmargin=5mm,
	% line-numbers
	numbers=left, % 
	numberstyle=\tiny\color{mygray}, % 
	% Code-highlighting	
	%identifierstyle=\color{red}, % 
	keywordstyle=\color{blue}, 
	stringstyle=\color{myStringRed}, % 
    commentstyle=\color{mygreen}, % 
    showspaces=false, % 
    showstringspaces=false, %    
	% Code
	language=HTML5,
	tabsize=2, % 
	extendedchars=true, % 
	breaklines=true, % 
	%float=hbp,% 
    %columns=flexible, %     
    breakautoindent=true, % 
    captionpos=b, %
    sensitive=false,  
	morecomment=[s]{<!--}{-->}, 
	morestring=[b]',
	% Klassen Namen aus Liquid
	emph=[1]{body, page, post, site, article, footer, links, header, sidebar, head, navigation, content},
	emphstyle=[1]{\color{myClassBlue}},
	% Anweisungen in Liquid
	emph=[2]{for, endfor, in, include },
	emphstyle=[2]{\color{mypurple}},
	%Attribute von Liquid
	emph=[3]{ html, posts, title},
	emphstyle=[3]{\color{mydarkgray}},
%	emph=[4]{px, fr, .5s},
%	emphstyle=[4]{\color{black}},	
	literate =%
	{Ä}{{\"A}}1
	{Ö}{{\"O}}1
	{Ü}{{\"U}}1
	{ä}{{\"a}}1
	{ö}{{\"o}}1
	{ü}{{\"u}}1
	{ß}{{\ss}}1
}


%Formatieren von JS Code
\lstdefinestyle{JS}{
	% General Design
	% Background
	backgroundcolor=\color{white}, % 	
	basicstyle=\ttfamily\scriptsize,	
	%frame=single, % 
	frame=trBL,
	framexleftmargin=5mm,
	% line-numbers
	numbers=left, % 
	numberstyle=\tiny\color{mygray}, % 
	% Code-highlighting	
	%identifierstyle=\color{red}, % 
	keywordstyle=\color{blue}, 
	stringstyle=\color{myStringRed}, % 
    commentstyle=\color{mygreen}, % 
    showspaces=false, % 
    showstringspaces=false, %    
	% Code
	language=JavaScript,
	tabsize=2, % 
	extendedchars=true, % 
	breaklines=true, % 
	%float=hbp,% 
    %columns=flexible, %     
    breakautoindent=true, % 
    captionpos=b, %
    sensitive=false, 
	% Klassen Namen 
	emph=[1]{Math, document, classList},
	emphstyle=[1]{\color{myClassBlue}},
	% Anweisungen
	emph=[2]{style, document},
	emphstyle=[2]{\color{mypurple}},
	% Methoden
	emph=[3]{min, setProperty, length, display, toggle, onclick, querySelector},
	emphstyle=[3]{\color{mydarkgray}},
	literate =%
	{Ä}{{\"A}}1
	{Ö}{{\"O}}1
	{Ü}{{\"U}}1
	{ä}{{\"a}}1
	{ö}{{\"o}}1
	{ü}{{\"u}}1
	{ß}{{\ss}}1
}


% Umbennung des Listings
\renewcommand\lstlistingname{\langlistingname}
\renewcommand\lstlistlistingname{\langlistlistingname}
\def\lstlistingautorefname{\langlistingautorefname}

% Abstände in Tabellen
\setlength{\tabcolsep}{\spaltenabstand}
\renewcommand{\arraystretch}{\zeilenabstand}
